\documentclass[12pt]{article}
\usepackage{amsmath}
\usepackage{graphicx}
\usepackage{hyperref}
\usepackage[latin1]{inputenc}
\usepackage[top=2cm, bottom=2cm, left=2cm, right=2cm, headsep=14pt]{geometry}
\usepackage[T1]{fontenc}
\usepackage[utf8]{inputenc}
\usepackage{spverbatim}
\title{Report 6 Gluster}
\author{Group 2}
% \author{Nguyen Vu Hung}
\begin{document}
\maketitle
  \section{Setup everything}
    \textbf{\large Install GlusterFS}\\
        Add the GlusterFS PPA 
        \begin{verbatim}
            sudo add-apt-repository ppa:gluster/glusterfs-3.13
        \end{verbatim}
        Update GlusterFS
        \begin{verbatim}
            sudo apt-get update
        \end{verbatim}
        Install GlusterFS
        \begin{verbatim}
           sudo apt-get install -y glusterfs-server
        \end{verbatim}
        
     \textbf{\large Create a trusted pool}\\
        Sometimes you have to configure the firewall before connecting. Use ifconfig to check your IP\\
        \begin{verbatim}
            sudo iptables -I INPUT -p all -s <ip-address> -j ACCEPT
        \end{verbatim}
            Note: When you input ip address, you do not need "< >" \\
            Add server to trusted pool
        \begin{verbatim}
           sudo gluster peer probe <ip-address>
        \end{verbatim}
        Check for all peer status to make sure you are connected
        \begin{verbatim}
           sudo gluster peer status
        \end{verbatim}   
        
    \textbf{\large Create a Distributed Replicated Volume}\\
        For server and each node make a storage to act as a brick. Do not grant the storage root permission\\
        For example
        \begin{verbatim}
            mkdir home/hung/gluster/test
        \end{verbatim}
        Create a distributed replicated volume
        \begin{spverbatim}
           sudo gluster volume create <volume name> replica <int> transport tcp <ip server1>:<path of brick server 1> <ip server2>:<path of brick server 2> ... force
        \end{spverbatim}
        \\Number of bricks is not a multiple of replica count <int>. For example\\
         \textbf{\small sudo gluster volume create test replica 2 transport tcp 
         192.168.0.160:/home/hung/gluster/test 192.168.0.175/home/lezardvaleth97/gluster/test force}
        \\Start the volume
        \begin{verbatim}
            sudo gluster volume start <volume name>
            sudo gluster volume start test
        \end{verbatim}
        \textbf{Setup clients}\\
        Each client have to install GlusterFS client
        \begin{verbatim}
            sudo apt-get install -y glusterfs-client
        \end{verbatim}
        Create a directory to mount glusterfs flies to client (use mkdir) then mount
        \begin{verbatim}
            mount -t glusterfs <host ip address>:/<volume name> <directory>
            mount -t glusterfs 192.168.0.160:/test /home/hung/glusterfs
        \end{verbatim}
        u can check if the system i mounted or not by
        \begin{verbatim}
            df -h
        \end{verbatim}
        From now, you should be able to transfer file from each node. Copy the file to the volume then it's done\\
        To check the volume status
        \begin{verbatim}
            sudo gluster volume info
        \end{verbatim}
 
    \section{Perform benchmark}
    To perform benchmark, first we send a brick with large size through gluster.\\
    The volume top command allows you to view the glusterFS bricks’ performance metrics
     \begin{spverbatim}
            sudo gluster volume top test read-perf bs 1000000 count 1 brick 192.168.0.175:/home/lezardvaleth97/gluster/test
        \end{spverbatim}
        or we can use
        \begin{verbatim}
            sudo iozone -r 128k -i 0 -i 1 -i 2 -t 24 -s 10G
        \end{verbatim}
        Set record size 128Kb, random write/rewrite, read/reread and write/read, set number of threads to 24 and specify the size of test at 10Gb.\\
        We can see that glusterfs transfer files faster with large size file. The reading speed of glusterfs is also faster than writing spped.
        Note: We used local area network to test.
        
    \section{Who does what}
Nguyen Vu Hung did and wrote the report
\end{document}
